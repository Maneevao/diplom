\likechapter{ВСТУПЛЕНИЕ}

Когда запускается персональный компьютер,
	одно из первых,
		что в нем происходит,
	- это загрузка встроенного программного обеспечения,
		которое находится в специальном модуле на материнской плате.
Это первый шаг,
	который задаётся программным способом,
	а также задаёт последующее поведение персонального компьютера.
Захватив управление этим модулем,
	появляется возможность вывода из строя всех программных механизмов защиты на компьютере,
	и достигается постоянное пребывание в системе,
		потому что этот компонент неразрывно связан с ней физически.
Поэтому этот компонент является приоритетным для злоумышлеников.

Существуют методы,
	которые защищают встроенное программное обеспечение на аппаратном уровне.
Они реализуют необходимый пласт защиты от большого множества угроз.
А теперь представим,
	что система не представлена физически.
Это виртуальные машины,
	которые часто используются для различных целей.
На них ставят сервера, поднимают критические системы, тестируют огромное количество программных продуктов.
Они обладают множеством полезных качеств для этого,
	но как можно понять - них нет аппаратных составляющих,
		а значит защита,
			которая реализовывалась для встроенного программного обеспечения,
		на виртуальных машинах не существует.

Сейчас возможности встроенного программного обеспечения растут,
	в некоторых даже пресутствует стек технологий TCP/IP,
		который позволяет выходить в сеть не используя механизмы операционных систем.
Новая технология - UEFI/BIOS,
	которая только начинает появляться,
	как и все новые системы не обладает должной защитой.
А уровень вхождения в неё намного ниже,
	чем в технологию прошлых лет - BIOS.
