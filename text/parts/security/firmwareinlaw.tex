\section{Встроенное программное обеспечение в законодательстве РФ}
Помимо указаного раньше классификатора ПО от минкомсвязи\cite{minkomsvyaz:klassificator} 
термин BIOS в правовых актах используюется только с тезисом,
	говорящем об установке пароля,
	для защиты от возможных изменений стандартных настроек.
Очень часто, подобная защита обходилась простым сбросом всех настроек BIOS с последующей настройкой минимально необходимых.

Более детально настройка базовой системы ввода-вывода расписана в приказе Федеральной Службы по Интеллектуальной Собственности\cite{FSIS:prikaz},
	регламентирующем инструкции по обеспечению режима секретности.
В частности в пунктах 6.8-6.10  говорится,
	что необходимы настройки,
		которые исключают нестандартные виды загрузки ОС,
опять же парольная защита,
и в случае если встроенные тесты прошли неудачно, исключается возможность работы на компьютерной системе.
На этом деятельность по настройке BIOS заканивается.

Более детальное рассмотрение с точки зрения безопастности BIOS преведено Федеральной Службой, ответсвенной в области технического регулирования.
Как объект возможного достижения НСД BIOS рассматривается в выписке Федеральной Службы по Техническому и Экспортному Контролю\cite{FSTEK:vipiska},
	описывающей базовую модель угроз безопасности персональных данных при их обработке в информационных системах персональных данных от 15.02.2008.

Согласно данной выписке, существует три группы угроз непосредственного доступа в операционную среду информационной системы персональных данных:
\begin{enumerate}
\item Угрозы, реализуемые в ходе загрузки ОС;
\item Угрозы, реализуемые после загрузки ОС;
\item Угрозы, реализация которых определяется тем, какая из прикладных программ запускается пользователем.
\end{enumerate}

Рассматриваемый в работе тип угроз относится к первой группе.

Так как данная модель угроз рассматривается Федеральной Службой,
	значит существуют методы и средства,
	возможно определённые подрядчики,
		занимающиеся анализом данной группы угроз.
Согласно статье\cite{Xakep:ChinaSpy}, преведённой в журнале, нацеленном на информационную безопасность, метод заключается в статическом поиске сигнатур,
	по которым делается вывод о состоянии безопасности.
Сложно найти информацию, которая бы подтверждала бы выводы автора статьи или опровергала их.

В любом случае, защита необходима не только системам по обработке персональных данных, а это значит, что необходимо существование методов, которыми бы могли воспользоваться большии слои населения, а также определенные классы компаний.
