\subsection{Dreamboot}
На данный момент, вопрос с изучением загрузочных вирусов может встать более остро.
С введением технологии UEFI/BIOS, а также созданием проекта Tianocore,
	существует возможность ознакомиться с основными составляющими встроенного программного обеспечения.
Также в открытом доступе необходимая документация по проекту.
Неудевительно, что данная область привлекает всё больше специалистов.

Компанией, занимающейся иновациями в сфере информационной безопасности, - QuarksLab создан загрузочный вирус под названием Dreamboot\cite{proj:Dreamboot}.
Информация о данном вирусе впервые прозвучала на конференции по информационной безопасности 2013 года под названием HITBSecConf.
Он был создан для того,
	чтобы показать насколько стоит остро вопрос информационной безопасности в современных системах встроенного программного обеспечения.

В системе контроля версии github находится исходный код,
	а также исполняемый файл,
		который возможно проверить на собственной машине.
Таким образом было дано подтверждение тому,
	что угроза со стороны загрузочных вирусов реальна.
