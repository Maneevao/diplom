\section{Возможная модель нарушителя}
Основной критерий отбора - это возможности потенциального нарушителя.

История говорит о том,
	что это могут быть различный спецслужбы,
		способные добавлять загрузочную закладку во встроенное программное обеспечение ещё на момент производства компьютерных систем,
	а также что это могут быть люди,
		способные использовать уязвимости,
	найденные в компьютерных системах.

Немногие способны загрузить свой модуль во встроенное программное обеспечение.
Сегодня можно встретить в сети интернет инструкции по модификации и внесению своих исполняемых модулей, которые способны сделать все желающий, но будут необходимы определённые навики или аппаратура, которая удовлетворяла бы определённым требованиям.

Усугубляет ситуацию то,
	что в основном компьютерные системы производятся в иностранных государствах.
Так например, существует информация о том, что основные компании производящие процессорные устройства вставляют аппаратную закладку в свои модули.
Или потенциально возможные изменения программного кода на заводе изготовителе.
Также государство США упрощяет юридическую процедуру обыска компьютеров в любой стране. \cite{news:USAcomp}

Таким образом складывается следующая модель нарушителя:
\begin{enumerate}
\item Специальные службы иностранных государств;
\item Криминальные группировки, обладающие повышенными возможностями;
\item Исследователи информационной безопасности;
\item Люди, обладающие инженерными навыками, а также ознакомленные с вопросом.
\end{enumerate}

Очевидно, что у нарушителя должны быть большие возможности для осуществленния атаки на встроенное программное обеспечение,
	но это легко компенсировать полученным результатом - неудаляемая без специальных технических средств загрузочная закладка, полный контроль над системой.
