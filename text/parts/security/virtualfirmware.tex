\section{Встроенное программное обеспечение виртуальных машин}
Почему то считается,
	что угроза со стороны BIOS маловероятна,
	но существуют различные события в истории,
		которые показывают обратное.
Например, публикации некоторых технологий Агенства Национальной Безопасноси США, от не безызвестного символа свободы, Сноудена, а также различные уязвимости, связанные с ноутбуками Mac от компании Apple, перепрошивка которых была возможна сразу после режима сна ОС\cite{Apple:Sleep}.
А также существуют исследования людей, которые выкладывают в открытый доступ свою работу, в которой расписывается, как внедряется загрузочная закладка в встроенное программное обеспечение. Один из таких SmmBackDoor\cite{SmmBackDoor}.

Также считается, что чтобы перезаписать встроенное программное обеспечение необходимо аппаратное подключение специального модуля, который бы выполнял данную деятельность.
После истории с компанией Apple, ноутбуки которых позволяли перепрошивать встроенное программное обеспечение после режима сна, после истории с компанией AMD, криптографические ключи которой утекли в сеть вместе с исходным кодом их BIOS, а также редким обновлением данной системы, сложно поверить, что всё так без облачно.
Тем более, новая технология UEFI намерена регулярно обновляться из-под ОС, а это значит, что будут искаться пути обхода различных защитных механизмов и наверняка будут найдены, т.к. это сравнительно молодая технология, безопасностью которой ещё предстоит заниматься.
При всём при этом, важно отметить и то, что данная работа производится с виртуальными машинами встроенное программное обеспечение, которые представляется в виде файла на жестком диске, то есть для перезаписи не нужно ничего, кроме устройства способного переносить файлы на жестком диске.

Также можно вспомнить, что BIOS не молодая технология и созданы различные аппаратные способы защиты, которые защищают его например от перезаписи.
Эти технологии теоретически невозможно обойти, поэтому основная защита уже присутствует на материнской плате.
Здесь опять же важно отметить тот факт, что данная работа проводится с виртуальными машинами, а следовательно вся защита может быть представлена только программным способом.

Но, по мнению автора, больше всего усугубляет проблему то,
	что количество специалистов,
		работающих в данной области ничтожно мало,
	в основном это сотрудники,
		которые непосредственно создают это встроенное программное обеспечение, и редкие энтузиасты.

В сети интернет очень сложно найти доступный материал по данной тематике.
А то, что предлогается, не всегда удобно использовать, либо обладает недостатком - некомпетентность.
