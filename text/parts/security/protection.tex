\section{Существующие методы защиты}

Незадолго до написания работы компанией Google в сервисе под названием VirusTotal была добавлена функция,
	которая позволяет статически анализировать встроенное программное обеспечение.
Основной показатель, который сейчас используется -
	наличие в образе исполняемых файлов операционной системы Windows,
	что не является доставерным индикатором.
На данный момент технология только появилась в списке функционала сервиса,
	поэтому следует ожидать её расширения.

Аналогом статического анализа могут служить утилиты для выделения структурных элементов из образа, например UEFITool, PhoenixTool, Intel Flash Image Tool и др.
С помощью них возможно не только просмотреть содержимое образа,
	а также присутствует возможность модификации отдельных составляющих.

Ещё один метод защиты от загрузочных вирусов предложен компанией Лаборатория Касперского совместно с компанией KraftWay - антивирус для UEFI,
	который должен обеспечивать защиту от загрузочных вирусов.\cite{proj:KaspAntiv}
Сложно предположить методику работы антивируса,
	но можно сказать, что с большой вероятностью код антивируса не будет записываться на носитель кода, составляющего встроенное программное обеспечение.
Может возникнуть ситуация,
	что закладка во встроенном программном обеспеченим прекратит работу антивирусного ПО.
