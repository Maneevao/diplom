\section{Встроенное программное обеспечение}
Термин «Встроенное программное обеспечение» не часто используется в повседневной жизни,
	чаще используется аналог,
		появившийся исторически - «Прошивка».
Поэтому во многих источниках литературы чаще встречается подобное название.
В законодателных актах используется понятие - «Встроенное программное обеспечение»,
	но нет достаточного определения.
Автор приводит собственное определение после того,
	как назовёт те,
		что приводятся в использованных источниках.

«Прошивкой (англ. Firmware, fw) называют содержимое энергонезависимой памяти компьютера или любого цифрового вычислительного устройства — микрокалькулятора, сотового телефона, GPS-навигатора и т. д., в которой содержится его микропрограмма.» \cite{wiki:firmware}

Для отнесения ПО в различные классы Минкомсвязи создал классификатор:
Приказ Минкомсвязи России от 31.12.2015 N 621
"Об утверждении классификатора программ
для электронных вычислительных машин и
баз данных".\cite{minkomsvyaz:klassificator}
Согласно нему существует отдельный раздел - «Встроенное программное обеспечение»,
	в котором один единственный класс - «BIOS и иное встроенное ПО» - 
		программы, хранящиеся в постоянной памяти.\cite{minkomsvyaz:klassificator}
Данный приказ позволит определить, 
	что является целью рассмотрения,
	чтобы в последствии не путать различные классы ПО.
В самой же работе будет рассмотрена только технология BIOS,
	а именно её расширение UEFI/BIOS.

По мнению автора, более подходящее для работы определение имеет следующий вид:

Встроенное программное обеспечение -
	программное обеспечение, исходный код которого хранится аппаратными средствами, не предназначенными для выполнения задач, поставленными перед устройством,
	а его деятельность направленна на начальную инициализацию аппаратного обеспечения.

Как можно заметить в данном определении у встроенного программного обеспечения выделено две черты:

\begin{enumerate}
	\item Оно хранится в независимой области;
	\item Его основная функция - начальная инициализация.
\end{enumerate}
\subsection{История термина "прошивка"}
Термин «прошивка» появился в 1960-х годах, когда в ЭВМ использовалась память на магнитных сердечниках. В постоянных запоминающих устройствах (ПЗУ) использовались Ш-образные и П-образные сердечники. Ш-образные сердечники имели зазор около 1 мм, через который и укладывался провод. Для записи двоичной «1» провод укладывался в одно окно сердечника, а для записи «0» — в другое. В сердечник высотой 14 мм укладывалось 1024 провода, что соответствовало 1К данных одного разряда. Работа выполнялась протягиванием провода вручную с помощью «карандаша», из кончика которого тянулся провод, и таблиц прошивки. При такой кропотливой и утомительной работе возникали ошибки, которые выявлялись на специальных стендах проверки. Исправление ошибок осуществлялось обрезанием ошибочного провода и прошивкой взамен него нового.

В начале 1970-х годов появились П-образные сердечники, которые позволяли использовать для прошивки автоматические станки. Прошивка выполнялась уже не в устройстве ПЗУ, а в жгутах по 64, 128 или 256 проводов. Прошиваемые данные вводились в станок с помощью перфокарт. На специальной оснастке жгуты снимались со станка, обвязывались нитками, и концы проводов распаивались на колодки. После этого жгуты укладывались в блок ПЗУ. Как при ручной прошивке, так и при работе на прошивочном станке требовалась аккуратность и хорошее зрение, поэтому на прошивке работали молодые девушки.

В 1980-х годах термин «прошивка» стал вытесняться понятием «прожиг», что было вызвано появлением микросхем ПЗУ с прожигаемыми перемычками из нихрома или кремния, однако при более новых технологиях «прожиг» вышел из употребления, а «прошивка» осталась.

\subsection{Лицензионное соглашение с потребителем}
Обычно, заключая договор с производителем материнских плат,
	пользователь подписывает лицензионное соглашание,
		которое запрещяет извлекать встроенное программное обеспечение, а также изучать его различными способами.
В юридическом плане не всегда существует возможность для изучения встроенного программного обеспечения.
На аппаратуре до технологии UEFI и даже на некоторых моделях с поддержкой данной технологии не редки случаи,
	что существует всего несколько обновлений,
		создание которых датировано несколько летней давностью.
Таким образом получается,
	что встроенное программное обеспечение не обновляется,
	в нём не закрываются различные уязвимости,
	бывает, что искусствено занижены аппаратные возможности.
Все эти вопросы сложно решить с помощью легальных методов.

Чаще всего в изучении данного вопроса помогают исследования,
	которые были произведены против лицензионного соглашения,
	либо утечки исходного кода,
		одна из которых произошла в компании American Megatrends Incorporated\cite{news:AMIinsyde}.

Способы, которыми фирмы-производители следят за сохранностью встроенного программного обеспечения\cite{wiki:firmware}:
\begin{enumerate}
\item Лицензионное соглашение с потребителем запрещает извлекать и изучать «прошивки» тем или иным способом;
\item Самовольная замена «прошивки» на другую («перепрошивка») обычно прекращает действие гарантийных обязательств фирмы;
\item Процедуры обслуживания и изменения режимов работы микропрограмм обычно не разглашаются и в лучшем случае известны только работникам фирменных сервисных центров.
\end{enumerate}

Но не смотря на такую строгость с лицензией на использование встроенного программного обеспечения,
	существуют проекты,
		которые являются открытыми,
		распространяются под лицензиями - BSD, GNU GPL, 
	а также исходный код которых находится в свободном доступе.

Список основных проектов:
\begin{enumerate}
\item Tianocore\cite{proj:Tianocore} - UEFI/BIOS, который рассматривается в работе;
\item OpenBIOS\cite{proj:OpenBIOS} - проект, нацеленый на замену проприетарного ПО;
\item SeaBIOS\cite{proj:SeaBIOS} - основной BIOS, используемый в QEMU (на котором производится работа)
\item и др.
\end{enumerate}

Данная работа будет выполнена с использованием виртуального интерпретатора QEMU, с встроенным программным обеспечением от Tianocore.

\subsection{SLIC}

Перед компаниями производящими ПО стоял вопрос о подтверждении лицензии пользователя.
Для этой цели было создано три компонента подтверждения лицензии,
	а именно таблица ACPI\_SLIC table(SLIC),
		в которой расположены OEM SLP и OEM certificate.

OEM (от англ. original equipment manufacturer — «оригинальный производитель оборудования») -
организация, продающая под своим именем и брендом оборудование, сделанное другими предприятиями.

Каждой организации выдаются ключ-лицензия (OEM SLP) и цифровой сертификат (OEM certificcate), 
	а информация об этом хранится в таблице (ACPI\_SLIC).

SLIC (от англ. software licensing description table) - 
таблица, в которой хранится информация о лицензировании ПО.

OEM SLP (от англ. system locked pre-installation - «код продукта OEM») - 
специальный 25 значный ключ-лицензия.

OEM certificate («Цифровой сертификат OEM») -
файл в формате XML с расширением *.xrm-ms. Выдаётся фирмой Microsoft каждому крупному производителю ПК.

Таким образом получается,
	что на персональном компьютере получится запустить только ту систему,
		владелец которой обладает ключом,
	исключая возможность нелицензионного использования ПО.

