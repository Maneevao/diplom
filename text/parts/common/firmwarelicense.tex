\subsection{Лицензионное соглашение с потребителем}
Обычно, заключая договор с производителем материнских плат,
	пользователь подписывает лицензионное соглашание,
		которое запрещяет извлекать встроенное программное обеспечение, а также изучать его различными способами.
В юридическом плане не всегда существует возможность для изучения встроенного программного обеспечения.
На аппаратуре до технологии UEFI и даже на некоторых моделях с поддержкой данной технологии не редки случаи,
	что существует всего несколько обновлений,
		создание которых датировано несколько летней давностью.
Таким образом получается,
	что встроенное программное обеспечение не обновляется,
	в нём не закрываются различные уязвимости,
	бывает, что искусствено занижены аппаратные возможности.
Все эти вопросы сложно решить с помощью легальных методов.

Чаще всего в изучении данного вопроса помогают исследования,
	которые были произведены против лицензионного соглашения,
	либо утечки исходного кода,
		одна из которых произошла в компании American Megatrends Incorporated\cite{news:AMIinsyde}.

Способы, которыми фирмы-производители следят за сохранностью встроенного программного обеспечения\cite{wiki:firmware}:
\begin{enumerate}
\item Лицензионное соглашение с потребителем запрещает извлекать и изучать «прошивки» тем или иным способом;
\item Самовольная замена «прошивки» на другую («перепрошивка») обычно прекращает действие гарантийных обязательств фирмы;
\item Процедуры обслуживания и изменения режимов работы микропрограмм обычно не разглашаются и в лучшем случае известны только работникам фирменных сервисных центров.
\end{enumerate}

Но не смотря на такую строгость с лицензией на использование встроенного программного обеспечения,
	существуют проекты,
		которые являются открытыми,
		распространяются под лицензиями - BSD, GNU GPL, 
	а также исходный код которых находится в свободном доступе.

Список основных проектов:
\begin{enumerate}
\item Tianocore\cite{proj:Tianocore} - UEFI/BIOS, который рассматривается в работе;
\item OpenBIOS\cite{proj:OpenBIOS} - проект, нацеленый на замену проприетарного ПО;
\item SeaBIOS\cite{proj:SeaBIOS} - основной BIOS, используемый в QEMU (на котором производится работа)
\item и др.
\end{enumerate}

Данная работа будет выполнена с использованием виртуального интерпретатора QEMU, с встроенным программным обеспечением от Tianocore.
