\section{BIOS, UEFI/BIOS}
BIOS (от англ. basic input/output system - «базовая система ввода-вывода») - 
набор микропрограмм,
реализующих API для работы с аппаратурой компьютера и подключёнными к нему устройствами.

UEFI (от англ. Unified Extensible Firmware Interface - "универсальный интерфейс расширяемой прошивки") создаётся для того,
чтобы заменить технологию,
	которая уже считается устаревшей.

Разработка спецификации программного обеспечения UEFI,
а также SDK (от англ. software development kit - «комплект средств разработки»),
	известного под названием edk2 (EFI developnment kit 2),
производится компанией Unified Extensible Firmware Interface Forum. 
А до этого разработка была начата компанией Intel Corporation,
	которой были созданы первые редакции стандарта.

На момент написания работы спецификация расположена в свободном доступе на официальном сайте разработчика под версией 2.6, насчитывая 12 предшественников до версии 2.0.
Помимо спецификации на саму систему UEFI возможно найти спецификации на ACPI, UEFI shell, UEFI Platform Initializatio, а также другие документы по данной технологии.

Основные производители UEFI/BIOS для персональных компьютеров:
\begin{enumerate}
\item Award Software International Inc.;
\item American Megatrends Incorporated;
\item Insyde Software.
\end{enumerate}

