\section{Динамический анализ ПО}
«Динамический анализ кода - анализ программного обеспечения, выполняемый при помощи выполнения программ на реальном или виртуальном процессоре (в отличие от статического анализа).» \cite{wiki:dinanalise}

Динамический анализ применяется в тех областях,
	где главный критерий - надёжность программы.
Данный подход к анализу позволяет выявить то,
	что сложно, либо невозможно понять с помощью статического подхода.
Статический подход не всегда отвечает,
	какую функциональность несёт программа.

Этапы динамического анализа:
\begin{enumerate}
\item Подготовка исходных данных;
\item Проведение тестового запуска программы и сбор необходимых параметров (Динамическое тестирование);
\item Анализ полученных данных.
\end{enumerate}


В данной работе будет выполнен запуск ПО,
	с последующим сбором данных,
		необходимых только для выполнения поставленной академической задаче для подстверждения концепта выполнения.

Принципы динамического тестирования:
\begin{enumerate}
\item Белый ящик - исследуются данные о программном коде;
\item Черный ящик - исследуются входные и выходные данные;
\item Серый ящик - подбор входных данных по известной структуре программы.
\end{enumerate}


В данной работе не задаются входные данные,
	а также не смотрится то, что получается в итоге.
Поэтому логично предположить, что работа проводится с белым ящиком.
