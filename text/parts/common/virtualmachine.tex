\section{Виртуальная машина}

Виртуальная машина - программная и/или аппаратная система,
	эмулирующая аппаратное обеспечение некоторой платформы,
	или, виртуализирующая некоторую платформу.

Использование данной технолонии помогает исследовать критические элементы,
	которые сложно исследовать на реальных системах.
Упрощает процесс воспроизведения ПО под различными платформами,
	а также существенно упрощает процесс отладки.
Так как нет необходимости использовать два устройства,
	соединять их с помощью различных средств,
	а с использованием определённых решений в сфере виртуализации,
	так и вовсе отпадает необходимость настройки базовых составляющих.

На сегодняшний момент множество компаний заинтересовано в виртуальных технологиях. Всё чаще можно встрерить сервер на виртуальной машине или найти работника, который выполняет на ней основную свою деятельность, т.к. это позволяет избежать множества проблем с безопасностью, а также управлением системы. 

Применение виртуальных машин:
\begin{enumerate}
\item Ограничение возможностей программ (Песочница);
\item Работа с различными архитектурами;
\item Разделение ресурсов сервера (запуск нескольких серверов на различных виртуальных машинах);
\item Тестирование и отладка систем.
\end{enumerate}


Применение виртуальных машин постепенно становится чем-то обыденным. 
А если виртуальные машины становятся такими популярными,
	то всё острее стаёт вопрос их безопасности.
Существуют направления в сфере информационной безопасности,
	в которых обсуждаются вопросы манипуляцией внутри виртуальной машины,
	возможность обойти её,
	способы определения типа виртуального интерпретатора, гипервизора.
У виртуальных машин, как и у реальных,
	существует возможность исполнения кода,
	заменяющего встроенное программное обеспечение.
По сути оно почти ничем не отличается от настоящего,
	кроме некоторых поправок,
		которые можно не рассматривать в первом приближении.

В работе используется эмулятор - QEMU\cite{proj:QEMU}.
