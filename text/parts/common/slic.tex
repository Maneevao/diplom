\subsection{SLIC}

Перед компаниями производящими ПО стоял вопрос о подтверждении лицензии пользователя.
Для этой цели было создано три компонента подтверждения лицензии,
	а именно таблица ACPI\_SLIC table(SLIC),
		в которой расположены OEM SLP и OEM certificate.

OEM (от англ. original equipment manufacturer — «оригинальный производитель оборудования») -
организация, продающая под своим именем и брендом оборудование, сделанное другими предприятиями.

Каждой организации выдаются ключ-лицензия (OEM SLP) и цифровой сертификат (OEM certificcate), 
	а информация об этом хранится в таблице (ACPI\_SLIC).

SLIC (от англ. software licensing description table) - 
таблица, в которой хранится информация о лицензировании ПО.

OEM SLP (от англ. system locked pre-installation - «код продукта OEM») - 
специальный 25 значный ключ-лицензия.

OEM certificate («Цифровой сертификат OEM») -
файл в формате XML с расширением *.xrm-ms. Выдаётся фирмой Microsoft каждому крупному производителю ПК.

Таким образом получается,
	что на персональном компьютере получится запустить только ту систему,
		владелец которой обладает ключом,
	исключая возможность нелицензионного использования ПО.
