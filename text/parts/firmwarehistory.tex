\subsection{История термина "прошивка"}
Термин «прошивка» появился в 1960-х годах, когда в ЭВМ использовалась память на магнитных сердечниках. В постоянных запоминающих устройствах (ПЗУ) использовались Ш-образные и П-образные сердечники. Ш-образные сердечники имели зазор около 1 мм, через который и укладывался провод. Для записи двоичной «1» провод укладывался в одно окно сердечника, а для записи «0» — в другое. В сердечник высотой 14 мм укладывалось 1024 провода, что соответствовало 1К данных одного разряда. Работа выполнялась протягиванием провода вручную с помощью «карандаша», из кончика которого тянулся провод, и таблиц прошивки. При такой кропотливой и утомительной работе возникали ошибки, которые выявлялись на специальных стендах проверки. Исправление ошибок осуществлялось обрезанием ошибочного провода и прошивкой взамен него нового.

В начале 1970-х годов появились П-образные сердечники, которые позволяли использовать для прошивки автоматические станки. Прошивка выполнялась уже не в устройстве ПЗУ, а в жгутах по 64, 128 или 256 проводов. Прошиваемые данные вводились в станок с помощью перфокарт. На специальной оснастке жгуты снимались со станка, обвязывались нитками, и концы проводов распаивались на колодки. После этого жгуты укладывались в блок ПЗУ. Как при ручной прошивке, так и при работе на прошивочном станке требовалась аккуратность и хорошее зрение, поэтому на прошивке работали молодые девушки.

В 1980-х годах термин «прошивка» стал вытесняться понятием «прожиг», что было вызвано появлением микросхем ПЗУ с прожигаемыми перемычками из нихрома или кремния, однако при более новых технологиях «прожиг» вышел из употребления, а «прошивка» осталась.
